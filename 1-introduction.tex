\section{Introduction}

Deep neural networks trained with back-propagation
learning~\cite{Backprop89} are a method of choice to solve complex
problems with sufficient data. Popular graph computation
engines~\cite{Theano,Torch7,MXNet,TensorFlow,PyTorch} offer high-level
abstractions for optimizing and executing deep neural networks
expressed as graphs of tensor operations. These frameworks make
transparent use of heterogeneous computing systems, leveraging
highly-optimized routines for individual operators.  While these
operators are sufficient for many applications, they fall short in a
number of instances.  Developing a novel type of layer or network
architecture incurs high engineering cost or performance penalty.
Even if a new layer may be expressed in terms of existing library
primitives, performance is often far from peak for two reasons: missed
optimizations across operators, and no tuning for its specific
size, shape and data flow~\cite{FBFFT15}.  Our work aims at addressing this
\emph{productivity gap}.\footnote{The ``700 layers'' is a reference to
  a seminal paper on programming languages: P.~J. Landin. The next 700
  programming languages. \emph{Communications of the ACM},
  9(3):157--166, 1966.}

In parallel to the software problem, a hardware race has begun, fueled by the
needs for energy-efficient computing. With Google's TPU~\cite{TPU17}
and Microsoft's Brainwave project~\cite{Brainwave17} on the bleeding edge,
many large tech companies are pursuing their own hardware.
At Google~I/O~2018, Turing-award recipient John Hennessy called
for fully rethinking our hardware, compilers and language support for
domain-specific properties~\cite{HennessyIO18}, citing orders of magnitude
speedup opportunities and power constraints caused by the advent of dark
silicon~\cite{DarkSilicon}.

With the increasing problem complexity and hardware limitations,
growing the size of manually
optimized libraries will not scale to future demands.  To address
these challenges, we present a novel \emph{domain-specific flow}
capable of generating highly-optimized kernels for tensor
expressions. It leverages optimizations across operators and takes
into account the size and shape of data.  The polyhedral framework of
compilation emerged as a natural candidate to design a versatile
optimization flow satisfying the needs of the domain and target
hardware. It has demonstrated strong results in domain-specific
optimization~\cite{Polymage,VOBLA,Baghdadi2015Pencil,Elango:2018:DDL:3211346.3211354},
expert-driven meta-programming~\cite{URUK,CHiLL,Clay}, embedding of
third-party library code \cite{DBLP:conf/pldi/KongVSFPS13}, and
automatic generation of efficient code for heterogeneous
targets~\cite{PlutoGPU,RStream,PouchetFPGA,PPCG2013,Baghdadi2015Pencil,Zinenko2018Spatial}.
We attempt to take the best of both worlds, defining a domain-specific
language rich enough to capture full sub-graphs of modern
Machine Learning (ML) models,
while enabling aggressive compilation competitive to native
libraries. In doing so, we may temporarily sacrifice some of the
performance of über-optimized large matrix multiplications (e.g., compared
to the recent Diesel polyhedral compiler \cite{Elango:2018:DDL:3211346.3211354})
while providing full automation and ML framework integration.
Note that there is no fundamental difficulty in combining both approaches,
recognizing and
linking external library kernels when appropriate,
as illustrated in \autoref{sec:matching}.

Our contributions are the following:
\begin{enumerate}
\item the Tensor Comprehensions (TC) Domain-Specific Language (DSL)
  with a tensor notation close to the mathematics of deep learning,
  with an emphasis on improving productivity while maintaining a
  direct lowering path to the intermediate representation of a
  parallelizing compiler for GPU acceleration;
\item an intermediate representation and Just-In-Time optimizing
  compiler based on the polyhedral framework, enabling complex program
  transformations and levels of automation unmatched by any other
  compiler for the acceleration of computational sub-graphs of neural
  networks;
\item coordinated optimization algorithms with integrated functional
  correctness, profitability modeling, domain and target
  specialization; we propose a layered approach, relying on integer
  linear programming and other polyhedral algorithms to address the
  core program optimization and synthesis challenges, while resorting
  to evolutionary algorithms as a higher level of control, to select
  high level strategies and fine-tune transformation parameters;
\item the transparent integration of our flow into
  PyTorch~\cite{PyTorch} and Caffe2~\cite{Caffe2}, providing the fully
  automatic synthesis of high-performance GPU kernels from simple
  tensor algebra.
\end{enumerate}

The TC flow is also portable to other ML frameworks with a few lines
of code. While our initial implementation focuses on Nvidia GPUs, the
core technology applies to other types of accelerators with shared or
partitioned
memory~\cite{RStream,PouchetFPGA,PPCG2013,YukiDistributed13}; these
include vector and SIMD accelerators, and also the generation of
computational patterns suitable for ASICs with systolic designs and
efficient storage management involving non-volatile memory
technologies.
